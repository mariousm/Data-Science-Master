\documentclass[a4paper, 11pt]{memoir}
\usepackage[spanish,es-tabla]{babel}
\selectlanguage{spanish}
\usepackage[utf8]{inputenc}
\usepackage{graphicx}
\nonzeroparskip
\usepackage[left=3.50cm, right=3.0cm, top=3.0cm, bottom=3.0cm]{geometry}

% DEFINICIÓN DE COMANDOS
% Capítulos
\chapterstyle{bianchi}
\newcommand{\capitulo}[2]{
	\setcounter{chapter}{#1}
	\setcounter{section}{0}
	\chapter*{#2}
	\addcontentsline{toc}{chapter}{#2}
	\markboth{#2}{#2}
}

% DEFINICIÓN DE LA PORTADA
% \title{PEC4 - Estado del arte}
% \author{Mario Ubierna San Mamés}
% \date{\today}
% \maketitle % dentro de document

% COMIENZO DEL DOCUMENTO
\begin{document}

	% PORTADA
	\begin{titlingpage}
		\centering
		{\includegraphics[width=1\textwidth]{img/logo.png}\par}
		\vspace{1cm}
		{\bfseries\LARGE Universitat Oberta de Catalunya\par}
		\vspace{1cm}
		{\scshape\Large Máster de Ciencia de Datos \par}
		\vspace{3cm}
		{\scshape\Huge PEC4 - Estado del arte \par}
		\vspace{3cm}
		{\itshape\Large Deep Learning \par}
		\vfill
		{\Large Autor: \par}
		{\Large Mario Ubierna San Mamés \par}
		\vfill
		{\Large \today \par}
	\end{titlingpage}

	\newpage

	% Tabla de contenidos
	\tableofcontents

	% Resumen
	\capitulo{1}{Resumen}
	
	% 1.1 Referencia del artículo
	\section{Referencia del artículo}
	El artículo seleccionado es "\emph{Deep Learning Based Automatic Video Annotation Tool for 
	Self-Driving Car}", cuyos autores son \emph{N.S.Manikandan, K.Ganesan}. La fecha de 
	publicación del artículo data del \emph{19 de abril del 2019}, y fue publiacada por el grupo
	 \emph{TIFAC-CORE in Automotive Infotronics} perteneciente al Instituto Tecnológico de Vellore \cite{manikandan_deep_2019}.
	
	% 1.2 Descripción de la temática
	\section{Descripción de la temática}
	El \emph{Deep Learning} es un concepto muy usado actualmente, ha tenido un crecimiento
	exponencial en los últimos años. Es tal el impacto, que se hace uso de este campo para 
	la detección de diferentes objetos encontrados durante la conducción de coches autodirigidos.
	
	% 1.3 Novedades que presenta el artículo
	\section{Novedades que presenta el artículo}
	
	% 1.4 Resumen de la parte experimental
	\section{Resumen de la parte experimental}
	
	% 1.5 Conclusiones
	\section{Conclusiones}

	% Bibliografia
	\bibliography{bibliografia.bib}
	\bibliographystyle{ieeetr}

\end{document}