\documentclass[a4paper, 11pt]{memoir}
\usepackage[spanish,es-tabla]{babel}
\selectlanguage{spanish}
\usepackage[utf8]{inputenc}
\usepackage{graphicx}
\nonzeroparskip
\usepackage[left=3.50cm, right=3.0cm, top=3.0cm, bottom=3.0cm]{geometry}

% DEFINICIÓN DE COMANDOS
% Capítulos
\chapterstyle{bianchi}
\newcommand{\capitulo}[2]{
	\setcounter{chapter}{#1}
	\setcounter{section}{0}
	\chapter*{#2}
	\addcontentsline{toc}{chapter}{#2}
	\markboth{#2}{#2}
}

% DEFINICIÓN DE LA PORTADA
% \title{PEC4 - Estado del arte}
% \author{Mario Ubierna San Mamés}
% \date{\today}
% \maketitle % dentro de document

% COMIENZO DEL DOCUMENTO
\begin{document}

	% PORTADA
	\begin{titlingpage}
		\centering
		{\includegraphics[width=1\textwidth]{img/logo.png}\par}
		\vspace{1cm}
		{\bfseries\LARGE Universitat Oberta de Catalunya\par}
		\vspace{1cm}
		{\scshape\Large Máster de Ciencia de Datos \par}
		\vspace{3cm}
		{\scshape\Huge PEC4 - Estado del arte \par}
		\vspace{3cm}
		{\itshape\Large Deep Learning \par}
		\vfill
		{\Large Autor: \par}
		{\Large Mario Ubierna San Mamés \par}
		\vfill
		{\Large \today \par}
	\end{titlingpage}

	\newpage

	% Tabla de contenidos
	\tableofcontents

	% Resumen
	\capitulo{1}{Resumen}
	
	% 1.1 Referencia del artículo
	\section{Referencia del artículo}
	El artículo seleccionado es "\emph{Deep Learning Based Automatic Video Annotation Tool for 
	Self-Driving Car}", cuyos autores son \emph{N.S.Manikandan, K.Ganesan}. La fecha de 
	publicación del artículo data del \emph{19 de abril del 2019}, y fue publiacada por el grupo
	 \emph{TIFAC-CORE in Automotive Infotronics} perteneciente al Instituto Tecnológico de Vellore \cite{manikandan_deep_2019}.
	
	% 1.2 Descripción de la temática
	\section{Descripción de la temática}
	El \emph{Deep Learning} es un concepto muy usado actualmente, ha tenido un crecimiento
	exponencial en los últimos años. Es tal el impacto, que se hace uso de este campo para 
	la detección de diferentes elementos encontrados durante la conducción de coches autodirigidos.

	Esta tarea es fundamental en los tiempos en los que los coches autodirigidos van adquiriendo
	más importancia con empresas como \emph{Tesla}.

	El objetivo de este artículo es mostrar cómo es la identificación de diferentes objetos, 
	la clasificación de los mismos, la detección de los carriles y el seguimiento de la 
	trayectoria de todos los elementos anteriores. 
	
	Para ello se hace uso de técnicas de \emph{deep learning} y herramientas de vídeo, 
	con el fin de capturar cada una de las imágenes en tiempo real y clasificarlas.

	% 1.3 Novedades que presenta el artículo
	\section{Novedades que presenta el artículo}
	Inicialmente cuando se empezó a desarrollar los coches autodirigidos, la detección de objetos
	se realizaba de forma manual y para ello se empleaban programas \emph{open source}. Realizar
	la detección de objetos de forma manual es muy costosa en términos temporales y económicos.

	Es aquí donde entra las novedades del \emph{deep learning} y en consecuencia de este
	artículo, hacer este proceso de anotación en tiempo real más barato y preciso que de forma
	manual.
	
	Para alcanzar el objetivo que se plantea en el artículo hay dos estrategias diferentes,
	hacer uso de métodos de anotación semi-automáticos o totalmente automáticos.

	Cabe destacar que hay proyectos similares tal y como se indican en el artículo, pero éstos
	en su mayoría tienen un enfoque de detección semi-automática. En el presente estudio se 
	busca ir un paso más allá, hacer que la tarea de detección sea completamente automática.

	Con el fin de lograr este punto, se hace uso de un trabajo previo de \emph{Zhujun Xiao}, el
	cual diseñó un sistema de generación de imágenes con autodetección.
	
	% 1.4 Resumen de la parte experimental
	\section{Resumen de la parte experimental}
	En cuanto a la detección de objetos se han usado tanto \emph{YOLO} como \emph{Retinanet}.
	El modelo ganador ha sido \emph{Retinanet}, consigue una mayor precisión, una mayor exactitud y
	una menor pérdida.

	En lo que respecta a las propiedades de clasificación se comparan los modelos \emph{VGG-19} y
	\emph{Retinanet}. El modelo seleccionado como mejor es \emph{Retinanet}, presenta una menor
	pérdida y una mayor precisión media tanto en el entrenamiento como en la validación.

	Por otro lado, para la detección de líneas se hizo uso de \emph{Udacity} y \emph{LaneNet}.
	\emph{Udacity} tuvo dificultades en la detección de líneas de carreteras asiáticas. Debido
	a esto se hizo uso de \emph{LaneNet}.

	Finalmente, para el siguimiento de objetos se realizó el estudio de \emph{Udacity} y 
	\emph{Deep SORT}. Fue este último el que consiguió mejores resultados, por lo que se utilizó
	este modelo en la versión final.
	
	% 1.5 Conclusiones
	\section{Conclusiones}
	En este artículo podemos ver el cómo diferentes algoritmos de \emph{deep learning} son capaces
	de realizar diferentes tareas de identificación sobre un vídeo en tiempo real.

	A la vista de los resultados obtenidos, se consigue una exactitud media del 83\%, es decir,
	todos los modelos en su conjunto son capaces de identificar y anotar de forma automática el 
	83\% de los objetos analizados, si la anotación manual es el 100\% solamente es un 17\% peor.

	Aunque hay una pérdida en la exactitud del 17\% respecto a la clasificación manual, el tiempo
	requerido es muchísimo menor. Para hacernos una idea, el entrenamiento con CPU tarda 43
	minutos en entrenarse, con GPU 2 minutos y medio, pero de forma manual se tarda 3060 minutos.

	En conclusión, aunque la exactitud obtenida para los modelos de \emph{deep learning} es menor,
	el tiempo necesario para realizar la misma tarea haciendo uso de redes neuronales artificiales 
	frene a una labor manual es muy diferente, es casi 1200 veces más rápido la solución
	que se presenta en este artículo que hacer la misma tarea de forma manual.

	Por lo tanto, vemos que sí que se produce un gran avance en esta área. El objetivo en líneas 
	futuras podría ser mejorar la exactitud obtenida en este proyecto.

	% Bibliografia
	\bibliography{bibliografia.bib}
	\bibliographystyle{ieeetr}

\end{document}